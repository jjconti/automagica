\documentclass[12pt,spinewidth=var{SPINE_WIDTH},coverwidth=var{COVER_WIDTH},coverheight=var{COVER_HEIGHT},flapwidth=0cm]{bookcover}

\usepackage[spanish]{babel}
\usepackage[utf8x]{inputenc}
\usepackage[T1]{fontenc}

\newbookcovercomponenttype{center rotate}{
    \parbox[t][\partheight][c]{\partwidth}{
        \begin{center}
             \rotatebox[origin=c]{90}{#1}
        \end{center}}}
\usepackage{contour,lipsum}
\contourlength{1pt}
\definecolor{lightbrown}{RGB}{176,88,0}
var{DEFINED_COLORS}
\colorlet{title}{var{TITLE_COLOR}}
\begin{document}

\begin{bookcover}

% Background color on the whole cover
\bookcovercomponent{color}{bg whole}{color=var{COVER_COLOR}}

% Remark 
\bookcovercomponent{center}{above front}{
    \color{blue}var{DESCRIPTION}}

% Picture on the front, behind the title
\bookcovercomponent{normal}{front}{
    \vspace{var{IMAGE_VSPACE}}
    \centering
    \includegraphics[width=var{IMAGE_WIDTH}]{var{IMAGE_PATH}}
    \vfill
    \includegraphics[width=var{LOGO_WIDTH}]{var{LOGO_PATH}}
    \vspace{5mm}}

% Text on the front cover
\bookcovercomponent{normal}{front}{
    \centering
    \vspace{var{AUTHOR_VSPACE}}
    \color{title}\Large\bfseries
    var{AUTHOR}
    \par
    \vspace{var{TITLE_VSPACE}}
    \color{title}\Huge\bfseries
    var{TITLE}}

% Text on the spine
\bookcovercomponent{center rotate}{spine}{
        \color{title}\large
        var{TITLE} \hspace{var{SPINE_INTERTEXT_SPACE}} var{AUTHOR}}

\bookcovercomponent{center}{spine}{
        \vfill
        \includegraphics[width=var{SPINE_LOGO_WIDTH}]{var{SPINE_LOGO_PATH}}
        \vspace{2mm}}

% Text on the back cover
\bookcovercomponent{normal}{back}{
    \centering
    \vspace{var{BACK_VSPACE}}
    \parbox{100mm}{\color{var{BACK_TEXT_COLOR}}var{BACK_TEXT}}
    \vfill
    %\parbox{100mm}{\color{var{BACK_TEXT_COLOR}}http://www.juanjoconti.com/automagica/}
    \vspace{5mm}}

% Text and picture on the front flap
\bookcovercomponent{normal}{front flap}{
    \centering
    \vspace{20mm}
    \parbox{75mm}{\color{var{BACK_TEXT_COLOR}}
Juanjo Conti es un escritor y programador nacido en Carlos Pellegrini, provincia de Santa Fe. Se graduó de Ingeniero en Sistemas de Información en la Facultad Regional Santa Fe, perteneciente a la Universidad Tecnológica Nacional, en el año 2008 y en 2012, obtuvo una maestría en la misma especialidad. Dedica su tiempo libre a la lectura y la escritura.
Ha publicado los libros de relatos \textit{La máquina de los cuentos} (2010), \textit{Los Caballeros de la Rosa} (2012), \textit{Santa Furia }(2014), \textit{La prueba del dulce de leche} (2014) y \textit{Carne de los dioses} (2016), y  la novela \textit{Xolopes} (2014).
Es el principal desarrollador del software libre Automágica.
Actualmente vive en la ciudad de Santa Fe con su esposa y trabaja como programador y consultor.

\vspace{1cm}


Blog: http://www.juanjoconti.com/\\
Mail: jjconti@gmail.com\\
Twitter: @jjconti\\

}}

% Text on the back flap
\bookcovercomponent{normal}{back flap}{
    \vspace{20mm}
    \includegraphics[width=50mm]{var{LOGO_PATH}}
    \parbox{75mm}{\color{var{BACK_TEXT_COLOR}}
\vspace{1cm}
Automágica es un software libre que toma los originales de un autor escritos en un procesador de texto y genera en forma automática los archivos de tapas e interiores en el formato necesario para que el libro sea enviado a imprenta.
Consta de una interfaz gráfica, en la que se pueden definir variables generales como el título, el nombre del autor, el tamaño de página, el tamaño de letra. Y otras más específicas como el color de la tapa, el ancho de lomo, el texto de la contratapa, el texto de las solapas  y la imagen de la cubierta.
También da la flexibilidad necesaria para que usuarios avanzados puedan realizar cambios de diseño no contemplados en la interfaz.
El programa corre en diferentes sistemas operativos y se encuentra en activo desarrollo.

\vspace{1cm}

Más información:\\
http://www.juanjoconti.com/automagica/
}}

\end{bookcover}

\end{document} 