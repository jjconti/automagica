\documentclass[11pt,twoside,openright]{book}

\usepackage[a5paper,hmarginratio={3:2},bottom=0.8in,top=0.8in]{geometry}

\usepackage{times}
\usepackage[Lenny]{fncychap}
%\usepackage[Sonny]{fncychap}
%\usepackage[Glenn]{fncychap}
%\usepackage[Rejne]{fncychap}   % center
%\usepackage[Bjarne]{fncychap}   % right
%\usepackage[Bjornstrup]{fncychap}   % right bold

\usepackage[spanish]{babel}
\usepackage[utf8x]{inputenc}
\usepackage[T1]{fontenc}
\usepackage{mathptmx}
\usepackage{etoolbox}
\usepackage[titles]{tocloft}
\usepackage{pdfpages}
\usepackage{courier}
\usepackage{soulutf8}
\usepackage{setspace}

\renewcommand{\cftchapleader}{\cftdotfill{\cftdotsep}}

% change the space before the titles
\makeatletter
\patchcmd{\@makechapterhead}{\vspace*{50\p@}}{\vspace*{0pt}}{}{}
\patchcmd{\@makeschapterhead}{\vspace*{50\p@}}{\vspace*{0pt}}{}{}
\makeatother

% change the space after the titles
\renewcommand{\DOTI}[1]{%
    \raggedright
    \CTV\FmTi{#1}\par\nobreak
    \vskip 10pt}
\renewcommand{\DOTIS}[1]{%
    \raggedright
    \CTV\FmTi{#1}\par\nobreak
    \vskip 10pt}


\title{TÍTULO}
\author{AUTOR}
\date{}



% Avoid widows and orphans
\widowpenalty=10000
\clubpenalty=10000

\begin{document}

\pagenumbering{gobble}

\pagestyle{plain}


% IF PRINT
%\includepdf{empty.pdf}
%\includepdf{empty.pdf}

\maketitle

\cleardoublepage

\thispagestyle{empty}
\noindent
Edición automágica, 2016.\\

\vspace{0.5cm}

\noindent
\emph{TÍTULO} lleva la licencia
\emph{Creative Commons Attribution - NonCommercial - ShareAlike 4.0 Iternational}.
Esto significa que podés compartir esta obra y crear obras derivadas
mencionando al autor, pero no ha\-cer un uso comercial de ella.

\vfill

\noindent
%Más información sobre este libro:\\
\\

\cleardoublepage


\renewcommand*\contentsname{Índice}

\tableofcontents

\cleardoublepage

\pagenumbering{arabic}

\spacing{1.1}





\chapter*{Sobre la existencia de los fantasmas} \addcontentsline{toc}{chapter}{Sobre la existencia de los fantasmas}





Una noche de verano en mi pueblo, cuando yo tenía quince años, salí con dos amigos, como hacíamos todos los sábados. Nos estábamos entusiasmando con el alcohol, por lo que el plan era sentarse en la mesa de un bar a ver quién aguantaba más. No queríamos ir a alguno de los bares de la calle principal porque siempre encontrábamos a algún amigo de nuestros padres y temíamos que nos delate. La cobardía, entonces, nos llevó a rumbear por barrios con calles menos asfaltadas y luminosas.

Hicimos base en un sucucho de la calle Mazzini cerca de la ruta, al que todos llaman “el bar de Alemandri”. Empezaba a llover, así que nos mandamos para una de las mesas del fondo. Mis laderos pidieron un porrón para compartir y yo, que no había aprendido a disfrutar de la amargura de la cerveza, pedí un “aperitivo” (en casa solía tomar un dedo de Gancia rebajado con soda). En esa ocasión, el mozo me sirvió un vaso repleto de Cinzano y al costado, clavó un pequeño sifón de soda con el que (según parecía, así era el ritual) tendría que ir completándolo a medida que tomaba. Recuerdo que le pregunté a ese viejo flaco y huesudo si eso se tomaba puro y, después de reírse de mí con ganas, dijo que así lo tomaban los hombres. Mis amigos se descostillaban a carcajadas.

Cuando ya había pasado una hora desde nuestra llegada y aún no había tomado dos dedos del brebaje, la puerta del boliche se abrió por la fuerza del viento. Uno de los parroquianos se levantó a cerrarla, pero, al intentar hacerlo, un pie se lo impidió. Al pie lo seguía el ser más desagradable que hasta ese momento había visto en mi vida. Habrá tenido unos sesenta y cinco años, lucía ropa vieja, o sucia, no sé, tenía el pelo largo, la piel se le veía grasosa y sus dientes estaban todos podridos. Se sentó en una mesa y pidió caña Legui. Algunos nos miramos, cómplices, y nos dijimos que no era de por allí.

Volvimos a nuestra charla sobre fútbol; el mono Navarro Montoya acababa de hacer una atajada espectacular en el único televisor del antro. Era una repetición de unos años atrás, pero la estábamos siguiendo como si fuese en vivo.

Nos volvimos a percatar de la presencia del hombre cuando oímos los gritos. El forastero se había trenzado con el dueño del bar en una acalorada discusión sobre la existencia de los fantasmas. Lo escuchamos a Alemandri contar la historia de una tapera en un campo cercano:

—Todos los domingos a la noche, se oye el chirrido de una soga bajando un balde en el aljibe. Lo curioso es que en ese campo ya no hay aljibe, sino bombas eléctricas. Una familia, que trabajaba allí y vivía en la tapera, contó que una vez, tras oír los ruidos, salieron a la noche y con una linterna iluminaron a un hombre ataviado con ropas del 1800 que sacaba agua con un balde de madera. A su lado, una mujer, supuestamente su esposa, lavaba la ropa y la colgaba para que se seque.

El forastero bufó con ganas, para que lo escuchen.

—¿Cómo pueden creer en esos cuentos?

—Ningún cuento, señor —y Alemandri pronunció con desdeño la palabra “señor”—. Esto que relato me lo contó un primo mío, vecino del campo donde aparecen las ánimas. Le digo más; en una ocasión llegaron al pueblo dos misioneros. La tapera estaba desocupada y el dueño se las ofreció para pasar la noche. Era domingo. A la madrugada encontraron a los hombres de Dios en la ruta haciendo dedo para irse.

—Habladurías —dijo el forastero e hizo un gesto con la mano en el aire, como tratando de espantar una mosca inexistente, para restarle validez a la historia que el otro contaba.

Entonces, el duelo, que hasta ese momento era solo de vozarrones, se convirtió también en un duelo de gestos.

—Escuchame, sabandija —y mientras hablaba, el dueño del boliche, que había dejado de tratarlo de usted, sacudía el dedo índice como quien sacude la fusta antes de pegarle al caballo—. Vos no vas a venir a mi establecimiento a decirme qué es verdad y qué no.

El sabandija lo apuntó con el mentón.

—Me imagino, entonces, que si tu primo es vecino ya habrás ido a la tapera un domingo a la noche a ver a los fantasmas.

El cantinero enmudeció primero y tartamudeó después.

—Bueno… es que yo los domingos a la noche tengo el boliche repleto y una excursión paranormal es un lujo que no me puedo permitir. Además… además… esos de aquella mesa también los vieron. Fueron en camioneta a cazar palomas al monte que está detrás de la tapera. Se les hizo la noche y cuando regresaban caminando, vieron la escena. Basta decir que dejaron la camioneta y volvieron al pueblo corriendo.

Los dos de la esquina asintieron en silencio y ahora sí, Alemandri recuperó el color, sacó pecho y empezó a mover la cabeza esperando que el pelilargo, el aceitoso, el de los dientes podridos, responda. No lo hizo. El ganador de aquella discusión volvió a tomar la palabra.

—Entonces, ¿usted no cree en los fantasmas?

—Yo no —dijo el forastero. Y tomando el último trago de caña, se evaporó ante nuestros ojos.





\chapter*{El hombre que soñó con su gato} \addcontentsline{toc}{chapter}{El hombre que soñó con su gato}





Un hombre terminó de cenar, lavó los platos y sacó a su gato al patio para luego irse a dormir.

Mientras dormía, soñó que su gato lloraba en la puerta; se levantó y lo dejó entrar para luego volverse a dormir.

Mientras soñaba que dormía, soñó que su gato lloraba en la puerta; se levantó y lo dejó entrar para luego volverse a dormir.

Mientras soñaba que soñaba que dormía, soñó que su gato lloraba en la puerta; se levantó y lo dejó entrar para luego volverse a dormir.

Mientras soñaba que soñaba que soñaba que dormía, soñó que su gato lloraba en la puerta; se levantó y lo dejó entrar para luego volverse a dormir.

La secuencia se repitió cien veces durante la noche.

Cuando se despertó a la mañana siguiente, casi se desmaya cuando vio lo que había en la cocina.





\chapter*{Barba} \addcontentsline{toc}{chapter}{Barba}





El 19 de diciembre de 1994, antes de acostarme, me miré en el espejo del baño. En el reflejo, un hombre con una barba frondosa, arbórea, selvática, me miraba. Los pelos se extendían en infinitas ramificaciones oscuras que me cubrían todo el rostro, dejando, apenas, ver los ojos. Ojos pardos y barba negra con algún destello rojizo. Me lavé los dientes y me acosté.

Cuando me desperté al otro día y fui al baño a lavarme la cara, un rostro rasurado, lustroso, brillante, me encandilaba desde el reflejo. Ya no había ni barba negra, ni destello rojizo. Sin embargo, el recuerdo del día anterior, con el rostro lobuno, estaba muy vivo en mi memoria. ¿Lo habría soñado? ¿Me habría despertado sonámbulo a rasurarme? Si hubiese sido al revés, podría concluir que estuve dormido varios meses, pero no fue así. ¿O sí? Estas y otras cosas me pregunté esa mañana. Recuerdo bien la fecha porque ese día cumplí diez años.





\chapter*{Joel} \addcontentsline{toc}{chapter}{Joel}





El primer recuerdo que tengo de Joel es en el club San Martín. Blanco y pecoso. Muy pecoso. Tiene una remera rayada y pantalones cortos por los que se deslizan sus piernas flaquitas. También blancas, pero no pecosas. Lo tengo agarrado de la remera rayada y con la otra mano sostengo un palito que en la punta tiene caca. Y yo tengo caca en mi remera blanca. No lo recuerdo claramente, pero supongo que Joel me debe de haber manchado. No recuerdo claramente ese punto, pero sí recuerdo un sentimiento de ojo por ojo.

Joel me dice que si lo mancho, me pega una piña. Alrededor, una veintena de chicos de nuestra edad. No me acuerdo si es un cumpleaños o una clase de gimnasia. Debe de ser un cumpleaños porque tenemos edad de primaria y recién en la secundaria empezamos a tener clases de gimnasia en los clubes del pueblo.

El club San Martín es un predio enorme con distintos sectores bien diferenciados: el edificio, la cancha de básquet, la pileta, la cancha de fútbol y el parque. Nosotros estamos en el parque, entre el quincho y una cancha de volley playero.

—Me manchás y te pego una piña.

La veintena de chicos alienta a que concrete el acto y, como tocándolo con una varita mágica, acerco lentamente el palito con caca hasta que hace contacto con su remera a rayas.

¡Pum! Me sienta de una piña y los ojos se me humedecen enseguida. Me quiero largar a llorar, pero aguanto como puedo. Los otros chicos empiezan a empujar a Joel, le gritan cosas, lo insultan, le dicen “testigo de Jehová”, confundiendo su religión, porque Joel es evangelista.

Lo siguiente que recuerdo es llegar a mi casa llorando y abrazar a mi mamá.





\chapter*{Naranjas para don Bordesio} \addcontentsline{toc}{chapter}{Naranjas para don Bordesio}





En invierno, después de juntar naranjas en el patio, doña Magdalena envía a su hija Florencia con una bolsa repleta de esos tesoros dulces para su vecino de enfrente, don Bordesio.

—¿Cómo vas a salir con los pelos así? —le dice la mamá y la sienta frente al espejo para peinar su cabello rubio y hacerle dos trenzas, una a cada costado.

En julio cumple catorce años. Florencia, con sus trenzas, cruza la calle. El sol de un día cálido de invierno le da de lleno en las piernas blancas que deja ver su jardinero rojo y las calienta.

Entra por la puerta de atrás, sin tocar timbre. La mosquitera rechina, da un golpe y rebota un poco para terminar cerrándose. En la cocina de don Bordesio, un ventilador de techo gira en cámara lenta y una brisa casi imperceptible pero reconfortante se le cuela por el cuello de la remera. En la habitación, a bajo volumen pero imposible de no notar, suena un viejo disco de jazz. La niña, o ex niña, la muchachita, lo conoce de memoria.

El hombre cano de tez morena y tantos lunares como un cielo estrellado está sentado en su sillón, con sus anteojos oscuros. Los de siempre. Reconoce su aroma, su fragancia, antes de que ella hable. Aunque esté mezclada con la de las flores de azahar. Cuando le dice que le trae el regalo de su madre, sonríe y la llama. Con manos ásperas pero movimientos suaves, le acaricia el rostro para saludarla.

Las paredes de la casa están descascaradas, pintadas de un amarillo oscuro que deja ver el color anterior bajo sus cicatrices. De una de las paredes cuelga una foto en blanco y negro. En la foto se ve a un hombre joven montado a caballo, alegre.

La muchacha, que aprendió el ritual de niña y no lo cuestiona, sabe perfectamente lo que ocurrirá. Ella le sacará los anteojos para descubrir un par de ojos blancos y muertos que él rápidamente cerrará y ella besará. Él le desabotonará el jardinero rojo, que caerá pesado junto a los pies de medias con puntilla y zapatitos de charol. Un dedo firme dibujará placeres en el algodón que solo toca la madre de Florencia cuando lava su ropa interior.

—Hoy podés bajarme la bombacha.

Don Bordesio lo hace con manos nerviosas hasta llegar al jardinero rojo. Florencia levanta un pie y luego el otro. Da un paso al frente y se sienta en la rodilla del hombre. Una pierna de cada lado. Esas piernas que unos minutos antes el sol calentaba, ahora tienen temperatura propia.

—Haceme caballito.

Don Bordesio empieza a mover la pierna hacia arriba y hacia abajo, rápido, haciendo toda la fuerza con el pie, como cuando de más chica la hacía jugar.

Tímidos sonidos salen de la boca de la niña, o ex niña, la muchachita, la casi mujer, que siente el par de manos deslizarse por la espalda, bajo la remera, y suben  hasta la hebilla, que se desprende y libera aún más su cuerpo.

Los sonidos ya no son tan tímidos y don Bordesio hace los coros de aquel canto sin dejar de hacerle caballito.

En invierno, después de juntar naranjas de sus árboles, doña Magdalena envía a su hija Florencia con una bolsa repleta de esos tesoros dulces para su vecino de enfrente, don Bordesio. En otoño, le manda pan casero o buñuelos. Naranjas, pan casero, buñuelos, chocolates, una torta, perejil, tomates, hojas de aloe, pantallas de lo que en verdad le está enviando.



\cleardoublepage

%IMPRENTA
%\includepdf{empty.pdf}

\hspace{0pt}
\vfill
\begin{center}
Maqueteado automáticamente utilizando \emph{Automágica}.
\bigbreak
http://www.juanjoconti.com/automagica/
\end{center}
\vfill
\hspace{0pt}
\end{document}
