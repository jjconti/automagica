\documentclass[11pt,twoside,openright]{book}

\usepackage[a5paper,hmarginratio={3:2},bottom=0.8in,top=0.8in]{geometry}

\usepackage{times}
\usepackage[Lenny]{fncychap}
%\usepackage[Sonny]{fncychap}
%\usepackage[Glenn]{fncychap}
%\usepackage[Rejne]{fncychap}   % center
%\usepackage[Bjarne]{fncychap}   % right
%\usepackage[Bjornstrup]{fncychap}   % right bold

\usepackage[spanish]{babel}
\usepackage[utf8x]{inputenc}
\usepackage[T1]{fontenc}
\usepackage{mathptmx}
\usepackage{etoolbox}
\usepackage[titles]{tocloft}
\usepackage{pdfpages}
\usepackage{courier}
\usepackage{soulutf8}
\usepackage{setspace}

\renewcommand{\cftchapleader}{\cftdotfill{\cftdotsep}}

% change the space before the titles
\makeatletter
\patchcmd{\@makechapterhead}{\vspace*{50\p@}}{\vspace*{0pt}}{}{}
\patchcmd{\@makeschapterhead}{\vspace*{50\p@}}{\vspace*{0pt}}{}{}
\makeatother

% change the space after the titles
\renewcommand{\DOTI}[1]{%
    \raggedright
    \CTV\FmTi{#1}\par\nobreak
    \vskip 10pt}
\renewcommand{\DOTIS}[1]{%
    \raggedright
    \CTV\FmTi{#1}\par\nobreak
    \vskip 10pt}


\title{El libro de la jungla}
\author{Juan José Conti}
\date{}

\hyphenation{a-gudo}\hyphenation{se-cretaria}

% Avoid widows and orphans
\widowpenalty=10000
\clubpenalty=10000

\begin{document}

\pagenumbering{gobble}

\pagestyle{plain}


% IF PRINT
%\includepdf{empty.pdf}
%\includepdf{empty.pdf}

\maketitle

\cleardoublepage

\thispagestyle{empty}
\noindent
Edición automágica, 2016.\\

\vspace{0.5cm}

\noindent
\emph{El libro de la jungla} lleva la licencia
\emph{Creative Commons Attribution - NonCommercial - ShareAlike 4.0 Iternational}.
Esto significa que podés compartir esta obra y crear obras derivadas
mencionando al autor, pero no ha\-cer un uso comercial de ella.

\vfill

\noindent
%Más información sobre este libro:\\
http://www.ellibrodelajungla.com\\

\cleardoublepage


\renewcommand*\contentsname{Índice}

\tableofcontents

\cleardoublepage

\pagenumbering{arabic}

\spacing{1.1}



\chapter*{Visita al dentista} \addcontentsline{toc}{chapter}{Visita al dentista} 
 Tengo doce años, o trece, o catorce. O tal vez, diez. Estoy en esa sala de
 espera de dos por uno, paredes blancas casi totalmente cubiertas de diplomas,
 asientos de cuerina que rechinan y revistero con ejemplares de la década
 anterior.
 
 Estoy solo, esperando. La sala tiene tres puertas. Una da a la calle y es por
 donde entré. La otra da al escritorio de la secretaria y la acabo de golpear.
 La tercera, desde la que viene un sonido agudo y eléctrico, el sonido de una
 pequeña rueda de piedra girando a muchas revoluciones, un sonido que se
 enciende y que se apaga a intervalos casi regulares, es la puerta que da al
 consultorio del dentista.
 
 Ahora se abre la segunda puerta, la de la secretaria. Liliana es joven y tiene
 rulos negros que siempre aparentan estar húmedos. Me dice que el doctor ya me
 va a atender, que cuando termine con la paciente que está viendo ahora, sigue
 conmigo. «Viendo». Un eufemismo.
 
 Tomo una de las revistas y paso, sin mirar, las páginas hasta llegar al final.
 Hasta las historietas. Hay una de Quino. Leo. Me río, pero no estoy seguro de
 que haya entendido del todo el chiste. Los chistes de Quino siempre tienen dos
 niveles. Uno superficial, con el que se puede reír hasta un chico de nueve años
 y otro, más profundo. Yo me concentro en esa otra parte. En tratar de entender
 el lado B del chiste. El lado que para ser comprendido necesita algo más que
 haber leído lo que dicen los personajes o haber visto el dibujo. A veces, se
 necesita que el lector haya leído cierto libro, o conozca tal pintura, o haya
 oído cierta noticia.
 
 Arranco la página con el chiste y me la guardo en el bolsillo de la campera.
 Estoy dejando la revista en su lugar cuando se abre la tercera puerta. Sale una
 mujer agarrándose la cara con la mano izquierda. Con la derecha, cierra tras de
 sí la puerta.
 
 ---Siguiente ---llaman con la voz de un dios mitológico unos segundos después.
 
 Abro la tercera puerta e ingreso a ese espacio luminoso en donde una multitud
 de herramientas filosas cuelgan expuestas, expectantes, esperando a ser
 utilizadas. Sin mediar palabra, me acuesto dócil en el asiento reclinado.
 
 Entonces me asalta un temor, una duda. ¿Me habrá visto el dentista cortar la
 hoja de la revista?
 
 Busco sus ojos. Los encuentro, penetrantes, ojos de dinamita que me miran
 detrás del barbijo. No necesito ver las muecas que pueda hacer con la boca. Con
 los ojos le alcanza para decirme que fue testigo del momento en que le robé la
 historieta y la guardé en el bolsillo de la campera.
 
 Intento levantarme pero no puedo. Estoy rodeado en su sillón por él en un
 flanco y por el brazo hidráulico que le sostiene las herramientas en el otro.
 
 Me unta la encía con una especie de gel y luego me apunta con una jeringa el
 lugar exacto que antes untó. Siento cómo la aguja se clava con fuerza y temo
 que la punta aparezca del lado de adentro de la boca. De la bandeja que
 sostiene el brazo hidráulico, toma un par de herramientas metálicas que
 tintinean entre sí y, en mi desesperación, en las notas producidas por el azar,
 descifro una melodía fúnebre.  Ya no tengo dudas. Sabe de mi fechoría y me va a
 torturar hasta que confiese.  Empieza con un gancho que parece un anzuelo de
 pescador y lo introduce entre dos muelas.
 
 ---¿Duele? ---me pregunta con sádico placer mientras escarba con el
 instrumento.
 
 Yo, que no puedo hablar por la anestesia, abro grandes los ojos y emito un
 sonido gutural para decirle que sí, que me duele.
 
 ---Vas a tener que aguantar ---me dice sin mirarme como única respuesta. Y con
 el pie acciona un interruptor que pone en funcionamiento una especie de torno
 de mano, su herramienta preferida. No puedo verlas, pero imagino que cuando el
 artefacto entra en contacto con mis dientes, una catarata de chispas saltan en
 el aire como si mi boca fuera la de un volcán en erupción.
 
 Trato de resistir. Clavo las uñas de las manos en los apoyabrazos de mi asiento
 y doblo los dedos de los pies tratando de mantenerme en la superficie de la
 Tierra. Nada es suficiente.
 
 Adivino una sonrisa de placer bajo el barbijo. Esa visión y el sonido de la
 rueda que gira sin parar rebotando contra mis dientes le ponen fin a mi
 resistencia. Exhalando, me doy por vencido y confieso.
 
 ---Fui yo, yo me robé el chiste de Quino ---. Pero las palabras no salen.
 Apenas un balbuceo que no logra atravesar las capas de saliva, sangre y dolor.
 
 La secretaria Liliana entra al consultorio y me ve tirado, sangrando, a medio
 camino entre la resistencia y la deshonra. Me doy vergüenza.
 
 Se acerca a donde estoy. Desde mi posición, solo puedo ver sus piernas. Hasta
 que se agacha y levanta algo del suelo.
 
 ---¿Esto es tuyo?
 
 Mi botín. La página con la historieta. A presencia de prueba, la confesión ya
 no es necesaria.
 
 ---¿Te gusta Quino? ---me pregunta el dentista, y se baja el barbijo para que
 le salgan mejor las palabras.
 
 Muerto de vergüenza ante la prueba que me condena, ya vencido, asiento con la
 cabeza.
 
 ---Si querés ---me dice---, cuando te vayas, fijate en el revistero. Creo que
 las revistas que están ahí tienen historietas de ese tipo. Llevate las que
 quieras. Y vos, Liliana, a ver si renovamos el catálogo, que ahí todavía hay
 revistas de cuando este era el consultorio de mi padre.
 


\chapter*{Sobre la existencia de los fantasmas} \addcontentsline{toc}{chapter}{Sobre la existencia de los fantasmas} 
 Una noche de verano en mi pueblo, cuando yo tenía quince años, salí con dos amigos,
 como hacíamos todos los sábados. Nos estábamos entusiasmando
 con el alcohol, por lo que el plan era sentarse en la mesa de un bar para ver
 quién aguantaba más. No queríamos ir a alguno de los bares de la calle
 principal porque siempre encontrábamos a algún amigo de nuestros padres
 y temíamos que nos delate. La cobardía, entonces, nos llevó a rumbear
 por barrios con calles menos asfaltadas y luminosas.
 
 Hicimos base en un sucucho de la calle Mazzini cerca de la ruta, conocido
 como «el bar de Alemandri». Empezaba a llover, así que nos mandamos para
 una de las mesas del fondo. Mis laderos pidieron un porrón para
 compartir y yo, que no había aprendido a disfrutar de la amargura de la
 cerveza, pedí un «aperitivo» (en casa solía tomar un dedo de Gancia
 rebajado con soda). En esa ocasión, el mozo me sirvió un vaso repleto
 de Cinzano y al costado, clavó un pequeño sifón de soda con el que
 (según parecía, así era el ritual) tendría que ir completándolo a medida
 que tomaba. Recuerdo que le pregunté a ese viejo flaco si eso se tomaba
 puro y, después de reírse de mí con ganas, dijo que así lo tomaban los
 hombres. Mis amigos se descostillaban a carcajadas.
 
 Cuando ya había pasado una hora desde nuestra llegada y aún no había 
 tomado dos dedos del brebaje, la puerta del boliche se abrió con fuerza 
 por el viento.
 Uno de los parroquianos se levantó para cerrarla, pero, cuando estaba por
 hacerlo, un pie se lo impidió. Al pie lo seguía el ser más desagradable
 que hasta ese momento había visto en mi vida. Tendría unos sesenta y cinco años,
 lucía ropa vieja, o sucia, no sé, tenía el pelo largo, la piel se le veía
 grasosa y sus dientes estaban todos podridos. Se sentó en una mesa y pidió
 caña Legui. Algunos nos miramos, cómplices, y nos dijimos que no era de por allí.
 
 Volvimos a nuestra charla sobre fútbol; el mono Navarro Montoya acababa
 de hacer una atajada espectacular en el único televisor del antro. Era
 una repetición de hace unos años, pero la estábamos siguiendo como si
 fuese en vivo.
 
 Nos volvimos a percatar de su presencia cuando escuchamos los gritos. El
 forastero se había trenzado con el dueño del bar en una acalorada discusión
 sobre la existencia de los fantasmas. Lo escuché a Alemandri contar la
 historia de una tapera en un campo cercano: 
 
 ---Todos los domingos a la noche
 se escucha el chirrido de una soga bajando un balde en el aljibe. Lo curioso
 es que en ese campo ya no hay aljibe, sino bombas eléctricas. Una familia,
 que trabajaba en ese campo y vivía en la tapera, contó que una vez, tras oír los
 ruidos, salieron a la noche y con una linterna vieron a un hombre ataviado
 con ropas del 1800 sacando agua con un balde de madera. A su lado, una mujer,
 supuestamente su esposa, lavaba la ropa y la colgaba para que se seque.
 
 El forastero bufó con ganas para que lo escuchen. 
 
 ---¿Cómo pueden creer en esos cuentos?
 
 ---Ningún cuento, señor ---y Alemandri pronunció con desdeño la palabra
 «señor»---. Esto que relato me lo contó un primo mío, vecino del campo donde
 aparecen las ánimas. Le digo más, en una ocasión llegaron al pueblo dos
 misioneros. La tapera estaba desocupada y el dueño se las ofreció para
 pasar la noche. Era domingo. A la madrugada, encontraron a los hombres de
 Dios en la ruta haciendo dedo para irse.
 
 ---Habladurías ---dijo el forastero e hizo un gesto con la mano en el aire,
 como tratando de tumbar una mosca inexistente para restarle validez a la historia que el otro contaba.
 
 Entonces, el duelo, que hasta ese momento era solo de vozarrones, se convirtió
 también en un duelo de gestos.
 
 ---Escuchame, sabandija ---y mientras hablaba, el dueño del boliche, que había
 dejado de tratarlo de usted, sacudía el dedo índice como quien sacude la
 fusta antes de pegarle al caballo---. Vos no vas a venir a mi establecimiento
 a decirme qué es verdad y qué no.
 
 El sabandija lo apuntó con el mentón. 
 
 ---Me imagino, entonces, que si tu
 primo es vecino ya habrás ido a la tapera un domingo a la noche a ver a
 los fantasmas.
 
 El cantinero enmudeció primero y tartamudeó después. 
 
 ---Bueno... es que yo
 los domingos a la noche tengo el boliche repleto y una excursión paranormal
 es un lujo que no me puedo permitir. Además... además... esos de aquella
 mesa también los vieron. Fueron en camioneta a cazar palomas al monte que
 está atrás de la tapera. Se les hizo la noche y cuando volvían caminando,
 vieron la escena. Basta decir que dejaron la camioneta y volvieron al
 pueblo corriendo.
 
 Los dos de la esquina asintieron en silencio y ahora sí,
 Alemandri recuperó el color, sacó pecho y empezó a mover la cabeza
 esperando que el pelilargo, el aceitoso, el de los dientes podridos,
 responda. No lo hizo. El ganador de aquella discusión volvió a tomar la
 palabra. 
 
 ---Entonces, ¿usted no cree en los fantasmas?
 
 ---Yo no ---dijo el forastero. Y tomando el último trago de caña, se evaporó
 ante nuestros ojos.
 
 


\chapter*{Juan} \addcontentsline{toc}{chapter}{Juan} 
 En el año 1993, yo estaba en tercer grado. Faltaba tiempo para que tenga mi
 primera computadora, pero ya me sentía atraído por esas máquinas blancas y
 luminosas, mezcla de electrodoméstico con máquina de escribir. Una vez a la
 semana teníamos clase de computación, tres alumnos por computadora. Uno
 manejaba el teclado, otro el mouse y el tercero miraba sobre los hombros de los
 otros dos.
 
 En mi pueblo había dos escuelas. Casi no conocía a los chicos que iban a la
 otra. Un domingo al mediodía, viendo el noticiero local, conocí a uno de esos
 desconocidos. Se llamaba Juan. El televisor lo mostraba sentado en una silla de
 ruedas; tenía los dedos doblados y hablaba con dificultad mientras se babeaba
 un poco. La entrevista se debía a que algún funcionario provincial le había
 regalado una computadora. Juan le mostraba a la periodista un juego en el que
 un pequeño dinosaurio se movía en la pantalla poniendo huevos explosivos. Lo
 odié. A Juan no parecía importarle ni la silla, ni sus dificultades, ni mi
 odio. En su cara, bajo un par de lentes grasosos, brillaba una enorme sonrisa.
 
 Cinco años después, empecé la secundaria. En mi pueblo, había un solo
 secundario. El primer día, nos juntaron a los de las dos escuelas, nos
 mezclaron y nos dividieron en dos cursos. Cuando me senté por primera vez en mi
 pupitre, al otro lado del salón, en la primera fila, sobre ruedas cromadas y
 con una \emph{notebook} del tamaño de mi carpeta, estaba Juan, con la misma
 sonrisa que había visto por televisión. Para ese entonces, yo ya tenía mi
 computadora en casa y me sabía unos cuantos trucos, pero una \emph{notebook}
 todavía era algo raro de ver. Así que fingí amistad para poder usarle la
 computadora en los recreos.
 
 Una mañana, la profesora de Lengua entró al salón con el semblante muy serio.
 Las otras profesoras, nos informó, se habían quejado de los muchos «horrores»
 ortográficos que tenía nuestro curso. Por lo tanto, desde ese día, los primeros
 veinte minutos de cada clase se iban a dedicar al dictado de palabras que luego
 ella corregiría.
 
 ---Cierren las carpetas, saquen una hoja y escriban ---dijo y empezó el dictado
 sin siquiera darnos tiempo a abuchear el anuncio---: Decisión, azotea, exento,
 agazapado, alférez, hosco, delgadez, exequias, habichuela, helénico, hinojo,
 ignífugo, ínfulas, levadizo, licencia.
 
 Luego de cada palabra, la profesora hacía una pausa para que Juan terminara de
 tipear. Escribía lento, a razón de una palabra por minuto. Apuntaba a cada
 letra con uno de sus largos dedos y luego dejaba caer todo su peso, cual
 péndulo humano. Volvía a tomar carrera y repetía el proceso. Luego punto y
 \emph{enter}.
 
 Cuando terminó el dictado, la profesora pasó por los bancos recogiendo las
 hojas y señalando los errores más comunes. Cuando llegó a la \emph{notebook} de
 Juan, adoptando la pose de una madre piadosa, se arrodilló junto a él, dejando
 su rostro a la altura de la pantalla. En voz baja, leyó lo que Juan había
 tipeado. Cuando se incorporó, estaba notoriamente emocionada.
 
 ---Ni un error ---dijo la profesora, marcando con fuerza la palabra «un»---.
 Ustedes... ---y bamboleó un índice acusador--- deberían aprender de él ---y con
 un rápido movimiento de brazo señaló a Juan--- que, a pesar de todas sus
 limitaciones, se esfuerza y estudia todo los días.
 
 En ese momento no pude contenerme más y, con un ruido estridente de sifonazo de
 soda, se me escapó de entre los dientes parte de la carcajada que venía
 reteniendo. Para mí, no era ningún secreto que Juan tenía activado el corrector
 ortográfico.
 
 ---¿De qué se ríe, alumno? ---me soltó la profesora con enojo, casi incrédula
 de que me estuviera riendo del pobre compañero.
 
 Las palabras se me ahogaban y de los ojos me caían lágrimas. En uno de los
 relinchos, titánico esfuerzo por dejar de reírme para adentro, lo vi a Juan que
 desde atrás de la profesora, con un movimiento lento, se llevaba el dedo índice
 de la mano derecha sobre los labios.
 
 No pude traicionarlo y estoicamente acepté el castigo de ser enviado a la
 dirección. Desde ese día, Juan y yo somos amigos.


\chapter*{Paredón} \addcontentsline{toc}{chapter}{Paredón} 
 Durante la escuela primaria, jugaba mucho al fútbol. Jugaba en los recreos de
 la escuela, en el club y en una canchita cerca de casa. A esa canchita iba en
 bici, una bici roja con calcomanías plateadas. La dejaba tirada y corría hasta
 donde el resto estaba pateando. Éramos siempre seis o siete. La mayoría, de la
 misma escuela; algunos un año más chicos, otros un año más grandes, y, de vez
 en cuando, algún que otro desconocido. Por lo general,  pensábamos que los
 desconocidos no iban a la escuela, a ninguna, y eso les daba un aura de
 renegados, un halo de misterio que provocaba admiración y respeto. Si la
 pateaban lejos, no les hacíamos buscar la pelota.
 
 La canchita quedaba al lado de la casa del Sapo; era de tierra y tenía arcos
 que su abuelo había hecho con ramas secas atadas con alambre. Nunca conoció la
 gramilla. Jugábamos tanto que no le dábamos tiempo al pasto a crecer. Era un
 manto marrón violáceo con textura de talco. Parecía que jugábamos en una cancha
 de chocolate «El Quilla».
 
 El Sapo era nuestro mejor delantero. Era zurdo y tenía una pegada que hacía
 temblar el alambrado que estaba atrás de los arcos. Muchas veces, si pegaba en
 el palo, hacía que el travesaño se cayera. Teníamos que suspender el partido y
 arreglar el arco. El resto éramos de regular para abajo. Ninguno iba a llegar a
 primera nunca. Sin embargo, el Sapo... sí podría haber llegado.
 
 Una característica que tenía nuestro grupo era que carecía de arquero. A nadie
 le gustaba tener que encargarse de defender el arco. Seguramente por eso, uno
 de los juegos más comunes entre nosotros era el «veinticinco». Alguien al azar
 empezaba en el arco. Cada vez que atajaba un tiro, podía usar la pelota para
 intentar golpear a uno de los otros jugadores. Si lo conseguía sin que la
 pelota pique, el jugador golpeado se convertía en el nuevo arquero. La
 secuencia seguía hasta llegar a los veinticinco puntos en el marcador global,
 que se acumulaban a través de los goles que íbamos haciendo. Diferentes tipos
 de goles valían diferente cantidad de puntos. Por ejemplo, un gol normal con el
 pie valía uno, pero un gol de cabeza, cinco. Cuando se llegaba a los
 veinticinco puntos, quien estaba en el arco tenía que cumplir una prenda. La
 clásica era el capotón: la víctima se agachaba, tratando de cubrirse, mientras
 todos los demás le pegábamos en la espalda.
 
 Una de esas tantas tardes, el Sapo estaba en el arco y el puntaje acumulado era
 de veinticuatro. En total, éramos seis chicos jugando. Los que no estábamos en
 el arco, nos acercábamos tocando la pelota y cuando estábamos lo
 suficientemente cerca, pateábamos con la esperanza de hacer un gol.
 Inmediatamente después, retrocedíamos a toda velocidad para intentar evitar la
 embestida del arquero, que nos arrojaba con potencia la pelota.
 
 En un determinado momento, Fitipaldi, uno que era nuevo en la escuela, pasó a
 tener la pelota. No se animó a patear y me dio un pase muy cerca del área. El
 Sapo se me vino arriba y pateé como pude. El Sapo atrapó la pelota en el aire y
 antes de que pudiera reaccionar, me la lanzó con suavidad sobre el cuerpo.
 Estaba tan cerca que no pude esquivarla y en el intento, me caí. Ante las
 carcajadas de todos, tomé mi lugar en el arco con la pelota en la mano. La suma
 seguía en veinticuatro.
 
 Todos estaban a más de quince metros de distancia. Imposible alcanzarlos. De
 todas formas, lo intenté. Con fuerza, lancé la pelota con la mano derecha,
 apuntando vagamente al montón. Se abrieron. Algunos para la derecha, otros para
 la izquierda, y la pelota pasó picando entre ellos.
 
 En menos de un minuto, luego de otro pase de Fitipaldi, el Sapo pateó al arco
 de media distancia y me hizo el gol que sumó veinticinco.
 
 ---¡Veinticinco, capotón! ---gritó uno mientras yo iba tomando la posición
 adecuada para recibir el castigo.
 
 ---No, mejor no. Tengo otra prenda ---dijo el Sapo y a continuación, explicó el
 «paredón»---. El que tiene que cumplir la prenda se para de espalda a nosotros,
 mirando la pared. Ponemos la pelota a seis pasos largos y uno patea «a fundir»
 para pegarle en la espalda.
 
 ---¡Sí! ---vitorearon todos con entusiasmo.
 
 Yo no dije nada; estaba pensando en mis posibilidades. A diferencia del capotón
 en el que tenía la golpiza asegurada, en esta nueva modalidad tenía la
 posibilidad de salir ileso. Por supuesto, si me embocaban, el golpe sería más
 duro.
 
 ---¿Y quién patea? ---preguntó uno.
 
 ---El que metió el último gol ---respondió el Sapo.
 
 Me ubiqué sin chistar y me persigné varias veces repitiendo en silencio una
 jaculatoria. «Que le erre», pensaba para mis adentros.
 
 El sonido pareció uno, pero en realidad fue un repique rápido de tambor: el
 botín del Sapo en contacto con la pelota de cuero y una fracción de segundo
 después, el cuero de la pelota contra mi espalda.
 
 ---¡Uhhhhhhh! --- gritó el coro de espectadores que, en una onomatopeya,
 expresó lo que yo no pude. Casi tampoco podía respirar. Encorvado y con un
 brazo levantado pedía clemencia.
 
 Como no quería que me vean llorar, agarré la bicicleta y pedaleando rápido me
 volví a mi casa. Estaba transpirado y sucio de tierra. Apenas llegué, sin
 saludar a nadie, me metí en el baño. Me saqué la remera y mirando para atrás,
 me vi la espalda en el espejo. Tenía un círculo rojo perfectamente delimitado.
 Mi espalda parecía la bandera de Japón. Ya no me dolía el golpe, pero ahora
 sentía un ardor en la piel. Me bañé y no hablé con nadie sobre el asunto.
 
 Al otro día, con el dolor y la vergüenza casi olvidados, volví a la canchita.
 Además de los de siempre, había un chico nuevo.  El Crema, me dijeron que se
 llamaba.
 
 Que hubiera uno nuevo era bueno. Todos nos complotábamos para hacerlo perder.
 Si no estaba en el arco, hacíamos que se acerque a él mediante falsas promesas
 de gol y lo traicionábamos a último momento. Cuando ya estaba en el arco,
 pateábamos de lejos sin importar que tardásemos en sumar. Así pasó esa tarde
 con el Crema. Cuando entró en el arco, no salió más.
 
 Veintidós.
 
 Veintitrés.
 
 Veinticuatro.
 
 Yo quería hacer el último gol. Y quería que le hagamos «paredón». Quería
 vengarme. No me importaba que el Crema no haya estado el día anterior. Solo
 quería darle un \emph{puntinazo} con fuerza a la pelota e incrustársela en la
 espalda.
 
 Veinticinco.
 
 Otra vez el Sapo, desde muy lejos, metió un zapatazo y cerró el juego.
 
 ---¿Qué eligen? ---dijo con la pelota bajo el brazo--- ¿capotón o paredón?
 
 ---¡Paredón! ---grité solo. Se ve que a los demás les daba lo mismo.
 
 El Crema, que conocía la prenda, se paró resignado mirando la pared.
 
 El Sapo depositó la pelota en el suelo y como si estuviera por ejecutar un
 penal,  se paró dos metros atrás. Miró al Crema, miró la pelota. Volvió a mirar
 al Crema y luego volvió a mirar la pelota. Tomó carrera y cuando su pie impactó
 en el esférico, yo tenía una sonrisa endiablada y me felicité por no estar ahí,
 de espaldas, esperando el impacto.
 
 La pelota no le dio en la espalda. Le pegó en la cabeza. El Crema se quedó
 tirado en el piso. Quieto.


\cleardoublepage

%IMPRENTA
%\includepdf{empty.pdf}

\hspace{0pt}
\vfill
\begin{center}
Maqueteado automáticamente utilizando \emph{Automágica}.
\bigbreak
http://www.juanjoconti.com/automagica/
\end{center}
\vfill
\hspace{0pt}
\end{document}